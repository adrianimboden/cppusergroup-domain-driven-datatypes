\section{Intro}

\begin{frame}[fragile]
	\frametitle{Intro}
	\begin{itemize}
		\item Wir haben genug Zeit
		\item Der Hauptteil wird Hands-On sein
		\item Diskussionen sind explizit erwünscht
	\end{itemize}
\end{frame}

\section{Motivation}
\begin{frame}[fragile,t]
	\frametitle{Motivation}
	\onslide<2>{\inputminted[firstline=3,lastline=11,framesep=5mm]{python}{../code/warum.py}}
	\onslide<1->{\inputminted[firstline=12,lastline=14,framesep=5mm]{python}{../code/warum.py}}
\end{frame}

\begin{frame}[fragile,t]
	\frametitle{Motivation}
	\only<1>{\inputminted[fontsize=\small,firstline=1,framesep=5mm]{java}{../code/warum.java}}
	\only<2>{\inputminted[fontsize=\small,firstline=1,framesep=5mm]{csharp}{../code/warum.cs}}
\end{frame}

\section{Basics}

\begin{frame}[fragile]
	\frametitle{Value Type}
	Alles was sich wie ein \enquote{int} verhält:
	\begin{itemize}
		\item Kopierbar
		\item Kein (sichtbares) sharing
		\item Keine Werte ausserhalb vom Wertebereich (nicht nullable)
		\item const ist transitiv
	\end{itemize}
	\pause{}
	Konkret:
	\begin{itemize}
		\item int
		\item std::string
		\item std::vector
		\item std::optional
		\item std::variant
	\end{itemize}
	\pause{}
	Kurz: fast alle Typen in der Standardlibrary
\end{frame}

\begin{frame}[fragile]
	\frametitle{Definitionsmenge}
	\begin{quote}
		In der Mathematik versteht man unter Definitionsmenge oder Definitionsbereich die Menge mit genau den Elementen, unter denen die Funktion definiert bzw. die Aussage erfüllbar ist.
	\end{quote}
	\begin{figure}
		\begin{minipage}{.3\textwidth}
			\includesvg[width=\linewidth]{img/injection.svg}
			\caption{wikipedia.org}
		\end{minipage}\hspace{10mm}%
		\pause%
		\begin{minipage}{0.6\textwidth}
			\inputminted[fontsize=\small,firstline=5,lastline=17,framesep=5mm]{cpp}{../code/src/01_intro.cpp}
		\end{minipage}%
	\end{figure}
\end{frame}

\begin{frame}[fragile]
	\frametitle{Datentyp für Definitionsmenge}
	\inputminted[fontsize=\small,firstline=5,lastline=17,framesep=5mm]{cpp}{../code/src/02_intro.cpp}
\end{frame}

\begin{frame}[t]
	\frametitle{Vergleich}
	\begin{figure}
		\begin{minipage}[t]{.3\textwidth}
			\inputminted[fontsize=\scriptsize,firstline=5,lastline=5,framesep=5mm]{cpp}{../code/src/01_intro.cpp}
			\inputminted[fontsize=\scriptsize,firstline=4,lastline=4,framesep=5mm]{cpp}{../code/src/01_intro.cpp}
			\inputminted[fontsize=\scriptsize,firstline=4,lastline=4,framesep=5mm]{cpp}{../code/src/01_intro.cpp}
			\inputminted[fontsize=\scriptsize,firstline=4,lastline=4,framesep=5mm]{cpp}{../code/src/01_intro.cpp}
			\inputminted[fontsize=\scriptsize,firstline=4,lastline=4,framesep=5mm]{cpp}{../code/src/01_intro.cpp}
			\inputminted[fontsize=\scriptsize,firstline=8,lastline=17,framesep=5mm]{cpp}{../code/src/01_intro.cpp}
			\vspace{1mm}
			\inputminted[fontsize=\scriptsize,firstline=19,lastline=22,framesep=5mm]{cpp}{../code/src/01_intro.cpp}
		\end{minipage}\hspace{20mm}%
		\pause{}
		\begin{minipage}[t]{.5\textwidth}
			\inputminted[fontsize=\scriptsize,firstline=20,lastline=24,framesep=5mm]{cpp}{../code/src/02_intro.cpp}
			\inputminted[fontsize=\scriptsize,firstline=4,lastline=4,framesep=5mm]{cpp}{../code/src/01_intro.cpp}
			\pause{}
			\inputminted[fontsize=\scriptsize,firstline=28,lastline=35,framesep=5mm]{cpp}{../code/src/02_intro.cpp}
			\pause{}
			\inputminted[fontsize=\scriptsize,firstline=4,lastline=4,framesep=5mm]{cpp}{../code/src/01_intro.cpp}
			\inputminted[fontsize=\scriptsize,firstline=4,lastline=4,framesep=5mm]{cpp}{../code/src/01_intro.cpp}
			\inputminted[fontsize=\scriptsize,firstline=37,lastline=40,framesep=5mm]{cpp}{../code/src/02_intro.cpp}
		\end{minipage}%
	\end{figure}
\end{frame}

\begin{frame}[t]
	\frametitle{Vergleich}
	\begin{figure}
		\begin{minipage}[t]{.3\textwidth}
			\vspace{27mm}
			\inputminted[fontsize=\tiny,firstline=3,lastline=21,framesep=5mm]{cpp}{../code/src/03a_intro.cpp}
		\end{minipage}\hspace{20mm}%
		\pause{}
		\begin{minipage}[t]{.5\textwidth}
			\inputminted[fontsize=\tiny,firstline=3,lastline=13,framesep=5mm]{cpp}{../code/src/03b_intro.cpp}

			\pause{}
			\inputminted[fontsize=\tiny,firstline=14,lastline=32,framesep=5mm]{cpp}{../code/src/03b_intro.cpp}
		\end{minipage}%
	\end{figure}
\end{frame}


\begin{frame}[fragile]
	\frametitle{Realbeispiele aus der Praxis}
	\begin{itemize}
		\item Utf8String
		\item Message
		\item Milliesconds
		\item Configuration
	\end{itemize}
\end{frame}

\section{Challenge für die, die wollen}
\begin{frame}[fragile]
	\frametitle{Beispiel Zahlen}
	\inputminted[fontsize=\tiny,firstline=5,lastline=30,framesep=5mm]{cpp}{../code/src/ausblick/01.cpp}
\end{frame}

\begin{frame}[fragile]
	\frametitle{Beispiel Zahlen Erweitert}
	\inputminted[fontsize=\tiny,firstline=55,lastline=63,framesep=5mm]{cpp}{../code/src/ausblick/02.cpp}
	\inputminted[fontsize=\tiny,firstline=65,lastline=65,framesep=5mm]{cpp}{../code/src/ausblick/02.cpp}
	\inputminted[fontsize=\tiny,firstline=67,lastline=67,framesep=5mm]{cpp}{../code/src/ausblick/02.cpp}
	\inputminted[fontsize=\tiny,firstline=69,lastline=69,framesep=5mm]{cpp}{../code/src/ausblick/02.cpp}
	\inputminted[fontsize=\tiny,firstline=71,lastline=72,framesep=5mm]{cpp}{../code/src/ausblick/02.cpp}
\end{frame}

\section{Hands-On}

\begin{frame}[fragile]
	\frametitle{Hands-on}
	\begin{figure}
		\includegraphics[width=0.6\linewidth]{img/hands_on.jpg}
		\caption{https://www.pinterest.com}
	\end{figure}%
\end{frame}
